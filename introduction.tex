\section{Introduction}
Let $X$ be a surface, by which we mean a nonsingular, two-dimensional integral scheme.  Let $\eta$ be its generic point, and $K$ its fraction field.  The short exact sequence associated to the sheaf of Cartier divisors on $S$ induces an injection of the Brauer group $\Br(X)$ of $X$ into the Brauer group of $K$.  Thus if $\Delta$ is a central simple $K$-algebra, it makes sense to ask if $\Delta$ is the image of an Azumaya algebra on $X$, ie. if there is an Azumaya algebra on $X$ whose restriction to $\eta$ is $\Delta$.  The answer can in general be ``no".  For example, if $X = \bbp^2_k$ for $k$ an algebraically closed field, then $\Br(X) = 0$, even though $\Br(K)$ is highly nontrivial.

However, $\Delta$ can always be viewed as the restriction of a maximal $\sheaf O_X$-order $A$ in $\Delta$.  Furthermore, such an $A$ is nicely behaved -- it is locally free as an $\sheaf O_X$-module of finite rank and is Azumaya on a dense open subset of $X$.  Moreover, the places where $A$ ramifies, ie. is not an Azumaya algebra, coincide with the cohomological ramification detected by the Artin-Mumford spectral sequence
$$0\rightarrow\Br(X)\rightarrow\Br(K)\rightarrow\bigoplus_{x\in X^1}H^1(\kappa(x),\bbq/\bbz)\rightarrow \bigoplus_{x\in X^2}\mu^{-1}\rightarrow\mu^{-1}\rightarrow 0,$$
as well as the ramification of the Brauer-Severi scheme $\BS(A)\xrightarrow{\pi}A$.

In this paper, we determine the local presentation of a maximal $\sheaf O_X$-order $A$ in $\Delta$ in the case that $\Delta$ is a quaternion algebra over $K$.  Using this, we provide an exposition of Artin and Mumfords result that for suitably chosen $S$ and $\Delta$, the Brauer-Severi variety $\BS(A)$ provides an example of a nonsingular nonrational unirational variety.

