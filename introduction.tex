\section{Introduction}
In this paper, we provide an exposition of an example by Artin and Mumford \cite{artin1972some} of a counter-example to Luroth's theorem for schemes of dimension $3$ over an algebraically closed field $k$.  The construction of the example relies on calcuations in both noncommutative algebra and algebraic geometry, for which it is interesting in its own right.  In particular, it demonstrates the principal that some higher dimensional schemes may be understood by more complicated algebra in a lower dimensional situation.  In our case, the lower dimensional data is a maximal order $A$ in a quaternion algebra on a rational surface $S$, and the higher dimensional scheme is the associated Severi-Brauer variety $\BS(A)$.

It is worth briefly outlining the strategy in the Artin and Mumford example, as it in particular serves to highlight the interplay between the commutative and noncommutative data to which we have vaguely gestured.  One of the lessons from the classical theory of maximal orders is that the structure of a maximal order over a DVR, especially a complete DVR, is very detailed.  Moreover, many properties of a reflexive order $A$ over a normal integral surface $S$ (maximal, Azumaya,...) are determined by the behavior of the localizations of $A$ at the codimension $1$ points of $S$.  The behavior of the localizations of $A$, in particular its ramification data, in turn fits into a long exact sequence relating the Brauer groups $\Br(S)$ and $\Br(K)$ of $X$ and its function field.

In their paper, Artin and Mumford are able to be even more precise by considering the behavior of maximal orders in quaternion algebras $\Delta$ over $K$.  If the ramification curve of $\Delta$ is nonsingular, then the local presentation of a maximal $\sheaf O_S$-order in $\Delta$ may be calculated explicitly.  Using this, explicit calculation of the fibres of the surjection $\pi: \BS(A)\rightarrow S$ are possible.  Moreover, one is able to show that for appropriately chosen ramification data the resultant Severi-Brauer scheme $\BS(A)$ is unirational but not rational.

