\section{A Nonrational Unirational Variety}
\subsection{Basic Definitions and Facts}
In this section, by ``variety" we mean an integral, separated scheme of finite type over an algebraically closed field $k$.  Recall that a variety $X$ is rational if it is birational to $\bbp^n_k$ for some $n$.  Equivalently $X$ is rational if its fraction field $K = K(X)$ is a purely transcendental extension of $k$ of finite transcendence degree.
\begin{ex}
The projective variety $\bbp^1_k\otimes\bbp^1_k$ is rational.
\end{ex}
\begin{ex}
A curve $C$ with genus $g>0$ cannot be rational.
\end{ex}

\begin{defn}
A variety $X$ is called \vocab{unirational} if for some $n$ there exists a dominant, generically finite, rational morphism $\bbp^n_k\xdashrightarrow{f}X$.
\end{defn}

\begin{thm}
Let $X$ be a variety with $\dim(X) = 1$.  If $X$ is unirational, then $X$ is rational.
\end{thm}
\begin{proof}
Suppose that $X$ is unirational.  Then the dominant rational morphism from $\bbp^1$ to $X$ extends to a morphism of schemes $\bbp^1\rightarrow X$.  This in turn induces a dominant morphism $\bbp^1\rightarrow \ol X$, where $\ol X$ is the projective closure of $X$.  Since $\ol X$ and $\bbp^1$ are both projective curves, the induced map $\bbp^1\rightarrow \ol X$ is in fact an isomorphism, and therefore $X$ is rational.
\end{proof}

\begin{thm}
Let $X$ be a variety with $\dim(X) = 2$ and assume that the characteristic of the base field $k$ is $0$.  If $X$ is unirational, then $X$ is rational.
\end{thm}
\begin{proof}
Suppose that $X$ is unirational.  Then one may argue that the the irregularity and $2$-pleurigenus of $X$ are both $0$, and therefore $X$ is raitonal by Castelnuovo's theorem.
\end{proof}

Given a variety $X$ over a field $k$ of characteristic $0$, if $X$ is unirational does it follow that $X$ is rational?  As it turns out, the answer is no.  For a cleverly chosen quaternion algebra $\Delta$ on a cleverly chosen rational surface $S$, the Brauer-Severi variety $\BS_2(A)$ of a maximal $\sheaf O_S$-order $A$ in $\Delta$ may be shown to be unirational but not rational.

\subsection{Construction}
In this section, we construct a surface $S$ with quotient field $K = k(x,y)$ and quaternion algebra $\Delta$ over $K$ whose ramification locus is a union of two nonsingular, disjoint curves $C_1,C_2$ on $S$ such that the Brauer-Severi scheme of a maximal order $\sheaf A$ of $\Delta$ over $S$ is unirational.  It will then follow from cohomological considerations that $\BS(\sheaf A)$ is a nonrational unirational variety.

Fix a nonsingular conic in $Z = V(\alpha)\subseteq \bbp^2_k$, and choose six distinct points $P_i^{(j)}\in Z$, $i=1,2$, $j=1,2,3$.
\begin{lem}
There exist nonsingular cubic curves $E_i = V(\beta_i) \subseteq \bbp^2_k$ such that
\begin{enumerate}[(a)]
\item  $E_i$ has double intersection with $Z$ at $P_i^{(j)}$ for $j=1,2,3$
\item  $E_i$ and $E_j$ transversally at nine distinct points $O_1,\dots,O_9$
\end{enumerate}
\end{lem}
\begin{proof}
See \cite{artin1972some}.
\end{proof}

Define $S$ to be the blow up of $\bbp_k^2$ at the intersection points of $O_1,\dots,O_9$ of $E_1$ and $E_2$, and let $C_i$ be the proper transform of $E_i$ for $i=1,2$.  Then $S$ is a simply connected rational surface, on which the nonsingular curves $C_1$ and $C_2$ are disjoint.  Note that since $S$ is rational, $\Br(S) = 0$.  Thus the Artin-Mumford sequence tells us that to obtain an element $\delta\in\Br(K)$ of order $2$ with $\ram(\delta/S)=C_1\cup C_2$, it suffices to specify a quadratic field extension of $K(E_i)$,  unramified over $C_i$, for $i=1,2$.

Let $q$ be the rational function on $\bbp^2_k$ whose divisor is $Z-2L$, with $L$ the line at infinity.  
\begin{lem}
The restriction $\ol q$ of $q$ to $E_i$ is not a square in $K(E_i)$ for $i=1,2$.
\end{lem}
\begin{proof}
Let $\sheaf L$ be the line bundle on $\bbp^2_k$ corresponding to the divisor $L$.  We claim that if $f\in \Gamma(E_i,\sheaf L|_{E_i})$, then $f$ has at most two zeros on $Z\cap E_i$.  To see this, let $f\in \Gamma(E_i,\sheaf L|_{E_i})$.  The natural map
$$\Gamma(Z,\sheaf L)\rightarrow\Gamma(E_i,\sheaf L|_{E_i})$$
is surjective, and therefore $f$ may be extended to a global section $\wt f$ of the sheaf $\sheaf L$.  Therefore $(\wt f) + L$ is a degree $1$ effective divisor, so we may choose a line $L'$ such that $L' = (\wt f) + L$.  It follows that the zero set of $f = \wt f|_{E_i}$ is exactly $L'\cap E_i$, and therefore the zeros of $f$ on $Z\cap E_i$ are contained in $Z\cap L'$, this latter set having only two elements.  This proves our claim.

Now suppose that $\ol q = f^2$ for some $f\in K(E_i)$.  Then $f$ has the same number of zeros on $Z\cap E_i$ as $\ol q$, namely $3$.  Furthermore, since $(q) = Z-2L$ we know that $(\ol q) + 2E_i\cdot L$ is effective, and therefore $(f) + E_i\cdot L$ is effective.  Hence $f\in \Gamma(E_i,\sheaf L|_{E_i})$.  By the argument of the previous paragraph, this is a contradiction.
\end{proof}

\begin{lem}
Let $L_i:=K(E_i)[\ol q^{1/2}]$ and let $E_i'$ be the normalization of $E_i$ in $L_i$.  Then $E_i'\rightarrow E_i$ is unramified.
\end{lem}
\begin{proof}
Since $E_i'\rightarrow E_i$ is locally of finite type, it suffices to show that for any point $P'\in E_i'$ with image $P\in E_i$, the induced map of residue fields $\kappa(P)\rightarrow\kappa(P')$ is a finite, separable field extension.  Then the local ring $R = \sheaf O_{E_i,P}$ is a DVR $(R,\nu)$ with residue field $k$ and fraction field $K(L_i)$.  Let $t$ be a uniformizer for $R$.  Depending on whether or not $P$ is one of the intersection points $P_i^{(j)}$ of $E_i$ and $Z$, the value of $\nu(\ol q)$ is either $0$ or $2$, meaning we may write $\ol q = ut^{2j}$ for some $u\in R^*$ and integer $j$.  Note that $R[u^{1/2}]$ is a semilocal Dedekind domain, hence a PID, and therefore integrally closed in its fraction field $L_i$.  Therefore since $u^{1/2}\in L_i$ is integral over $R$, we have $\sheaf O_{E_i',P'} = R[u^{1/2}]$.  Furthermore $R[u^{1/2}] = R[x]/(x^2-u)$ is unramified since $u$ is not a square in $R$.
\end{proof}
Thus there exists a unique two-torsion element $\delta\in\Br(K)$ whose ramification locus is $C = C_1\cup C_2$ and whose ramification data on $C_i$ is the nowhere-ramified extension $L_i$ for $i=1,2$.

Of course, this does not mean that $\delta$ is representable by a quaternion algebra: the two-torsion of $\Br(K)$ is only generated by quaternion algebras.  However, in our situation we may show that $\delta$ is split by the quadratic extension $L=K[q^{1/2}]$ of $K$, and therefore that $\delta$ is representable by a quaternion algebra.

The main idea of the argument is to construct a nonsingular $k$-scheme $Y$ with fraction field $L$, along with a birational morphism $Y\xdashrightarrow{f} S$, such that the induced map $f^*: \Br(L)\rightarrow\Br(K)$ sends $\delta$ to a Brauer class with trivial ramification data.  Then since $Y$ is rational, the Artin-Mumford sequence tells us $f^*(\delta)$ is trivial, and therefore that $\delta$ is split by $L/K$.
\begin{prop}
The Brauer class $\delta$ is split by the quadratic extension $L = K[q^{1/2}]$ of $K$.  Consequently $\delta$ may be represented by a quaternion algebra $\Delta$.
\end{prop}
\begin{proof}
We construct a nonsingular scheme $Y$ as follows.  Choose a cubic curve $E_0 = V(\beta)\subseteq \bbp^2_k$ such that the divisor $E_0$ pulls back on $Z$ to the divisor $\sum_{ij}P_i^{(j)}$.  Then $\beta^1\beta^2/\beta^2\in \Gamma(Z,\sheaf O) = k$, and therefore after rescaling $\beta$ we have $Z\subseteq V(\beta_1\beta_2-\beta^2)$.  Therefore there exists a homogeneous polynomial $\gamma$ of degree $4$ satisfying $4\alpha\gamma = (\beta_1\beta_2-\beta^2)$.

Let $Y_0$ be the quartic in $\bbp^3_k$ defined by
$$Y_0 = V(\alpha(X_0,X_1,X_2)X_3^2 + \beta(X_0,X_1,X_2)X_3 + \gamma(X_0,X_1,X_2))\subseteq \bbp^3_k.$$
Then the projection $\bbp^3_k\diff\{Q_0\}\rightarrow\bbp^2_k$ defines a birational morphism $Y_0\xdashrightarrow{f_0}\bbp^2_k$, for $Q_0 = [0:0:0:1]$.  Note that $f: Y_0\diff\{Q_0\}\rightarrow\bbp^2_k$ is a double cover of $\bbp^2_k$ ramified at $V(\beta^2-4\alpha\gamma) = V(\beta_1\beta_2) = E_1\cup E_2$.  In particular, the singular points of $Y_0$ different from $Q_0$ must occur at ramification points, and therefore correspond to the singular points of $E = E_1\cup E_2$, ie. the points $Q_i$ mapping to the intersection points $O_i$ of the curves $E_1$ and $E_2$.  Hence the $Q_i$ are exactly the singular points of $Y_0$.  Let $Y$ be the blow-up of $Y_0$ at $Q_0,\dots, Q_9$.  Then $Y$ is a nonsingular scheme with fraction field $L$, and the birational morphism $Y_0\xdashrightarrow{f_0} \bbp^2_k$ induces $Y\xdashrightarrow{f} S$.

We claim that $f^*(\delta)$ is trivial.  To prove this, note that since the ramification behavior of a Brauer class is determined in codimension $1$, blowing up at a handful of points does not change affect anything.  More specifically, blowing up a normal scheme at a collection of points induces an isomorphism of the Brauer groups of the function fields.  Therefore to show that $f^*\delta=0$, it suffices to show $f_0^*\delta=0$.  Away from the $Q_i$'s, $f_0$ is an etale double cover, and therefore if $y\in \ram(f_0^*\delta/Y_0)$ with $y\neq Q_i$, we must have $f_0(y)\in\ram(\delta/\bbp^2_k)$.  It follows that $\ram(f_0^*\delta/Y_0) = f_0^{-1}(\ram(\delta/\bbp^2_k))$.

Let $e_i$ be the generic point of $E_i\subseteq\bbp^2_k$.  Then $f^{-1}(e_i)$ consists of two codimension $1$ points $y_{i1},y_{i2}$ whose local rings $\sheaf O_{Y_0,y_{ij}}$ have fraction field $L_i$.  Hence the induced maps $f^*: H^1(\kappa(x_i),\mu_2)\rightarrow H^1(\kappa(e_i),\mu_2)$ kill the ramification data.  Note that for all $y\in Y_0^1$, $\rho(f^*\delta)_y = f^*(\rho(\delta)_{f(y)}).$  In this way, we see that the ramification data of $f^*\delta$ over $Y_0$ is trivial, and since $Y_0$ is rational it follows that $f^*\delta$ is $0$.  This completes the proof.
\end{proof}

\begin{lem}
Let $\Delta$ be the quaternion algebra defined above, and let $\sheaf A$ be a maximal $\sheaf O_S$-order in $\Delta$.  Then the Brauer-Severi variety $\BS(\sheaf A)$ of $\sheaf A$ is unirational.
\end{lem}
\begin{proof}
Using the same $f: Y\rightarrow S$ as above, we have that
$$\BS(\sheaf A)\times_S Y\cong \BS(f^*\sheaf A) \cong \bbp^1_Y.$$
Projection onto the first factor then gives us a surjection $\bbp^1_Y\rightarrow \BS(\sheaf A)$, and since $Y$ is rational this proves that $\BS(\sheaf A)$ is unirational.
\end{proof}
