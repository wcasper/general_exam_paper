\section{Azumaya Algebras}
\subsection{Twisted Forms and Nonabelian $H^1$}
\begin{defn}
Let $X$ be a scheme.  We define a \vocab{category of algebraic structures} on $X$ to be a subcategory $\catc$ of the category of sheaves on $X$.
\end{defn}
We will be primarily interested in case where $\catc/X$ is one of the following examples
\begin{itemize}
\item the category of $\sheaf O_X$-modules on $X$
\item the category of $\sheaf O_X$-algebras on $X$
\item the category of schemes over $X$
\item the category of $G$-sets, for $G$ a group scheme on $X$
\end{itemize}
Note that if $Y\in \Sch_X$, then pullback determines from $\catc$ a category of algebraic structures $\catc/Y$, and a right exact functor $\catc\rightarrow\catc/Y$.
\begin{defn}
Let $c\in \catc$.  We define an etale (fppf) \vocab{twisted form of $c$} to be an object $c'\in \catc$ such that there exists an etale (fppf) cover $\mathfrak U = \{f_i: U_i\rightarrow X\}$ with $c',c$ isomorphic in $\catc/U_i$ for all $i$.  We denote by $\twist(c)$ the set of isomorphism classes of twisted forms of $c$.
\end{defn}

The basis for understanding and classifying etale (fppf) twisted forms of $c\in C$ comes from a canonical identification of twisted forms of $c$ as \v{C}ech cohomology classes of the sheaf of automorphisms of $c$.  We will describe this correspondence here for etale twisted forms of $c$; the correspondence for fppf twisted forms is similar.  For $c\in C$, let $\Aut_{et}(c)$ denote the sheaf on the big etale site of schemes over $X$, defined by
$$\Aut_{et}(c): (f: Y\rightarrow X)\mapsto \Aut_{\catc/Y}(f^*c).$$
Then for a twisted form $c'$ of $c$, the local isomorphisms $\varphi_i: f_i^*c'\rightarrow f_i^*c$ generate a \v{C}ech cocycle $(\psi_{ij})\in \wc{C}^1(\mathfrak U,\Aut_{et}(c))$, for
$$\psi_{ij}: (p_i^*\varphi_i)\circ(p_j^*\varphi_j)^{-1}\in \End(f_{ij}^*c),$$
with $f_{ij}: U_i\times_X U_j\rightarrow X$ the induced map of the fiber product, and $p_i: U_i\times_X U_j\rightarrow U_i$ the canonical projection map.  The natural map $\wc{C}^1(\mathfrak U,\Aut_{et}(c))\rightarrow \wc{H}^1(\mathfrak U,\Aut_{et}(c))\rightarrow\wc{H}^1(X_{et},\Aut_{et}(c))$ completes the construction.  A descent argument then proves the following lemma
\begin{lem}\label{twisted form lemma}
Let $\catc$ be a category of algebraic structures on a scheme $X$, and let $c\in C$.  Then the canonical association of a \v{C}ech cocycle to a twisted form $c'$ of $c$ defines an injection
$$\twist(c)\hookrightarrow \wc{H}^1(X_{et},\Aut_{et}(c)).$$
A similar statement holds for the fppf twisted forms of $c$.
\end{lem}
\begin{proof}
See Milne \cite{milne1980etale}.
\end{proof}

\begin{ex}
Let $X = \spec(k)$, let $\catc$ be the category of schemes over $X$, and let $c = \spec(L)$ for $L$ a finite separable field extension of $k$.  If $Z\rightarrow X$ is etale, then $Z = \amalg_{i=1}^n\spec(L_i)$ where each $L_i$ is a finite, separable extension of $k$.  Therefore $Y\times_X Z = \amalg \spec(L_i\otimes_k L)$, and since $\Aut_{L_i}(L_i\otimes_k L)\cong \Aut_k(L)$, and thus
$$\Aut(c): Z\mapsto \Aut_{\Sch_Z}(Y\times_X Z) = \prod_{i=1}^nG,$$
where $G$ is the Galois group of the extension $L/k$.  Thus $\Aut(c)$ may be identified with the constant sheaf $G$ on the (big) etale site of $X$.  Consequently, $\wc{H}^1(X_{et},\Aut_{et}(c)) = \wc{H}^1(X_{et},G)$, and so twisted forms of $c$ are in one-to-one correspondence with $G$-torsors on $X$; for $X = \spec(k)$, these are exactly the Galois extensions of $k$ with Galois group $G$.
\end{ex}

In the previous example, we swept under the rug the issue of whether $\twist(c)\rightarrow\wc{H}^1(X,\Aut_{et}(c))$ is surjective.  Since descent theory may always be applied to ``glue together" the descent data coming from a representative $\wc{H}^1(X,\Aut_{et}(c))$ to a sheaf, the question is really one of representability: when is the sheaf that we glue together actually a member of the category $\catc$?  For example, it is clear that descent will be effective for $\catc$ the category of $G$-torsor sheaves.  However, whether it is effective for the category of $G$-torsors is equivalent to asking whether any $G$-torsor sheaf is representable.  This question in general can be difficult to answer.

Nevertheless, in many important instances the above map is an equivalence of categories.
\begin{ex}
Let $\catc$ be the category of quasi-coherent $\sheaf O_X$-modules, and let $c = \sheaf O_X^n$.  Then $\Aut(c)$ is equal to the group scheme $\bbgl_n$ defined on the big fppf site of $X$ by
$$\bbgl_n: (U\rightarrow X)\mapsto \bbgl_n(\Gamma(U,\sheaf O_U)).$$
Suppose $c'\in\catc$ is a twisted form of $c$.  Then $c'$ is fppf locally free of rank $n$.  By a follow-your-nose type argument, $c'$ is therefore Zariski locally free.  Thus $\twist(c)$ is the set of isomorphism classes of locally free rank $n$ $\sheaf O_X$-modules.  Since fppf descent is  effective for quasicoherent sheaves of $\sheaf O_X$-modules, we conclude that
$$\wc{H}^1(X_{fppf},\bbgl_n)\cong \twist(c) = \{\text{iso. classes of locally free, rank $n$ $\sheaf O_X$-modules}\}.$$
As a special case of this, we see that $\wc{H}^1(X_{fppf},\bbg_m) = \wc H^1(X_{fppf},\bbgl_1)$ classifies isomorphism classes of invertible sheaves on $X$, ie. $\wc{H}^1(X,\bbg_m) \cong \Pic(X)$.
\end{ex}
\begin{ex}
Let $\catc$ be the category of $\sheaf O_X$-algebras, and let $c = M_n: (U\rightarrow X)\mapsto M_n(\Gamma(U,\sheaf O_U))$.  The etale twisted forms of $c$ are exactly the degree $n$ Azumaya algebras on $X$.  Since etale descent is effective for $\sheaf O_X$-algebras, it follows that $\wc{H}^1(X_{et},\Aut_{et}(c))$ classifies isomorphism classes of degree $n$ Azumaya algebras on $X$.

What is $\Aut_{et}(c)$?  Conjugation provides a natural map $\bbgl_n\rightarrow \Aut_{et}(c)$ with kernel $\bbg_m$.  A version of the Skolem-Noether Theorem says that for any local ring $R$, all automorphisms of the ring $M_n(R)$ are inner.  Therefore, we see that the map of etale stalks $\bbgl_n\rightarrow\Aut_{et}(c)$ is surjective on stalks, and hence surjective.  In particular, we have a short exact sequence
$$0\rightarrow\bbg_m\rightarrow\bbgl_m\rightarrow\Aut(c)\rightarrow 0.$$
The cokernel of $\bbg_m\rightarrow \bbgl_n$ is exactly the group scheme $\bbpgl_n$, so we have an isomorphism $\Aut(c)\cong\bbpgl_n$.  Hence $\wc H^1(X_{et},\bbpgl_n)$ classifies isomorphism classes of Azumaya algebras on $X$.
\end{ex}

\begin{ex}
Let $\catc$ be the category of schemes over $X$, and let $c = \bbp^{n-1}_X$.  Then $\Aut(c) = \bbpgl_n$ and we have a canonical injection $\twist(c)\hookrightarrow \wc{H}^1(X_{fppf},\bbpgl_n)$.  As it so happens, this map is also surjective \cite{milne1980etale}.  Therefore $\wc{H}^1(X,\bbpgl_n)$ also classifies twisted forms of $\bbp^{n-1}_X$, so called Severi-Brauer varieties.
\end{ex}

Our last example describes a sort of universal interpretation of the elements of $\wc{H}^1(X,G)$ as isomorphism classes of $G$-torsor sheaves on $X$.  From some perspectives, this could be considered the right definition of the first nonabelian cohomology group.
\begin{ex}
Let $G$ be a group scheme on $X$.  By a (right) $G$-set on $X$, we mean a sheaf $F$ on $X$, and a morphism of sheaves $F\times G\rightarrow F$ compatible with the canonical morphism $G\times G\rightarrow G$, ie. so that the natural map $(F\times G)\times G\rightarrow F\times (G\times G)$ factors over $F$.  Let $\catc$ be the category of $G$-sets on $X$ with morphisms given by $G$-equivariant natural transformations.  Let $c=G$, with the canonical $G$-action by multiplication on the right.  We call $c$ the \vocab{trivial} $G$-torsor.

If $c'$ is an fppf-twist of $G$, then $c'$ is an fppf locally isomorphic to the trivial $G$-torsor: this is exactly the definition of a $G$-torsor sheaf on $X$!  Thus $\twist(c)$ is exactly the set of isomorphism classes of locally trivial $G$-torsor sheaves on $X$.  For any $Y\rightarrow X$, the set of $G$-equivariant maps $G\rightarrow G$ are exactly determined by where they send the identity.  Hence there is a canonical identification $\Aut(c) = G$.  Thus $\wc H^1(X_{fppf},G)$ is classifies the isomorphism classes of $G$-torsors on $X$.  We denote the set of isomorphism classes of $G$-torsors on $X$ as $\torsor(G)$.
\end{ex}

\begin{remk}
The first cohomology group is known to agree with the standard cohomology group for $G$ abelian on any site, so from now on we will leave off the check symbol, and simply write $H^1(X_{site},G)$.  Moreover, for many of the groups we work with, the first cohomology groups for the etale and fppf sites will agree -- in such situations, we will simply write $H^1(X,G)$ to mean cohomology with respect to this site.
\end{remk}

\subsection{Gerbes and Nonabelian $H^2$}
In this section, let $\catc$ and $\catb$ be categories, and $\varphi: \catc\rightarrow\catb$ a functor between them.
\begin{defn}
Suppose that $b\in\catb$.  We define the \vocab{fiber of $b$}, denoted $\varphi^{-1}(b)$, to be the category whose objects objecs in $\catc$ sent to $b$ under $\varphi$ and whose morphisms are morphisms sent to $\id_b$ under $\varphi$.
\end{defn}

\begin{defn}
Consider two objects $c,c'$ of $\catc$ and a morphism $f:c'\rightarrow c$, and set $b = \varphi(c), b'=\varphi(c')$ and $g = \varphi(f):b'\rightarrow b$.  We call $f$ \vocab{Cartesian} if for all $x\in \varphi^{-1}(b')$, and for all $h\in \Hom_{\catc}(x,c)$ satisfying $\varphi(h) = g$ there exists a unique morphism $f_0\in\Hom_{\varphi^{-1}(c')}(x,c')$ satisfying $ff_0 = h$.
\end{defn}

\begin{defn}
Let $\varphi: \catc\rightarrow\catb$ be a functor between categories $\catc$ and $\catb$.  We call the data $(\catc,\catb,\varphi)$ a \vocab{fibered category} if the following two properties hold
\begin{enumerate}[(i)]
\item  given $b',b\in\catb$, $c\in\varphi^{-1}(b)$ and $g: b'\rightarrow b$, there exists $c'\in\varphi^{-1}(b')$ and $f:c'\rightarrow c$ with $\varphi(f) = g$.
\item  the composition of two Cartesian morphisms is Cartesian
\end{enumerate}
In this case the object $c'$ is unique up to canonical isomorphism and is called the \vocab{pullback of $c$}, denoted $g^*c$.
\end{defn}

\begin{ex}
The identity functor $\id_{\catb}:\catb\rightarrow\catb$ defines $\catb$ as a fibered category over iteself.
\end{ex}

\begin{ex}{Grothendieck Construction}
Let $F: \catb\rightarrow \Cat$ be a pseudofunctor between a category $\catb$ and a $2$-category $\Cat$.  A construction attributed to Grothendieck allows us to use $F$ to build a fibered category $(\int F,\catc, \varphi)$.

Define a category $\int F$ whose objects are pairs $(b,c)$ with $b\in\catb$ and $c\in F(b)$, where
$$\Hom_{\int F}((b,c),(b',c')) = \{(g,\alpha) : b\xrightarrow{g}b',\ F(g)(c)\xrightarrow{\alpha}c'\}.$$
This category has a natural functor $\varphi: \int F\rightarrow \catb$ defined by $(b,c)\mapsto b$ and $(g,\alpha)\mapsto g$, and one may verify that $(\int F,\catc,\varphi)$ is a fibered category.
\end{ex}

\begin{defn}
Let $\catc\rightarrow\catb$ and $\catc'\rightarrow\catb$ be fibered categories, and let $F:\catc'\rightarrow\catc$ be a $1$-morphism of fibered categories over $\catb$.  Then we define the \vocab{inertia category of $\catc'$ over $\catc$} $\inertia_{\catc'/\catc}$ to be the category whose objects are pairs $(c',f)$ with $c\in \catc'$ and $f:c'\rightarrow c'$ an automorphism satisfying $F(f) = \id_{F(c')}$, and whose morphisms $(c_0',f_0)\rightarrow (c_1',f_1)$ are morphisms $h: c_0'\rightarrow c_1'$ such that $hf_0 = f_1h$.  We define the \vocab{inertia category of $\catc$} to be $\inertia_{\catc/\catb}$, and denote it by $\inertia_{\catc}$
\end{defn}

Now consider the case when the base category $\catb$ is a site.
\begin{defn}
Let $(\catc,\catb,\varphi)$ be a fibered category, with $\catb$ a site.  We call this fibered category a \vocab{prestack} if for any $b\in\catb$, and $a,a'\in\varphi^{-1}(b)$ the functor $F:\catc/b\rightarrow\Sets$ defined by
$$(f:b'\rightarrow b)\mapsto \Hom(f^*a',f^*a)$$
is a sheaf.
\end{defn}

\begin{defn}
Let $(\catc,\catb,\varphi)$ be a fibered category, with $\catb$ a site.  Let $\mathfrak U = \{g_i: b_i\rightarrow b\}$ be a covering of an object $b\in\catb$.  Then there is a natural map from $\varphi^{-1}(b)$ to descent data on $\mathfrak U$.  If all descent data on $\mathfrak U$ arises in this way, then we say \vocab{descent is effective on $\mathfrak U$}.  If descent is effective for every choice of $\mathfrak U$ and $b$, we say descent is effective on $\varphi$.  A prestack on which descent is effective is called a \vocab{stack}.
\end{defn}

We will mostly be interested in the case that the base category $\catb$ is a site on the category of schemes over $X$, ie. $\catb = X_{\text{site}}$.  For simplicity, we will refer to a stack $(\catc,X_{\text{site}},\varphi)$ as a stack on $X$, and denote it by $\varphi: \catc\rightarrow X_{\text{site}}$ or just by $\varphi:\catc\rightarrow X$, if the site structure on $X$ is clear from context.

\begin{ex}
Let $F$ be a sheaf on $X$, ie. a functor $F:\Sch_X^{\text{op}}\rightarrow\Sets$.  Then the Grothendieck construction gives a stack $\int F\rightarrow X$ on $X$.
\end{ex}

\begin{defn}
Let $\varphi: \catc\rightarrow X$ be a stack on $X$.  The stack is called a \vocab{gerbe} if it satisfies the following additional properties
\begin{enumerate}[(i)]
\item  for all $b\in\catb$ the fiber $\varphi^{-1}(b)$ is a groupoid (ie. all morphisms are isomorphisms)
\item  for every $b\in \catb$, there exists a covering $\mathfrak U = \{b_i\rightarrow b\}$ such that each $\varphi^{-1}(b_i)$ is nonempty
\item  for any $b\in\catb$, and $a,a'\in \varphi^{-1}(b)$, there exists a covering $\mathfrak U=\{f_i: b_i\rightarrow b\}$ of $b$ such that $f_i^*a$ and $f_i^*a'$ are isomorphic for all $i$
\end{enumerate}
\end{defn}

\begin{defn}
Let $G$ be a group scheme on $X$, and let $\catc\rightarrow X$ be a gerbe on $X$.  Consider the natural projection $p: \inertia_{\catc}\rightarrow X$.  We call $\catc$ a \vocab{$G$-gerbe on $X$} if $p^{-1}$ is naturally isomorphic to $G$ as sheaf of groups on $X$.
\end{defn}

\begin{ex}
Let $G$ be a group scheme on $X$.  We define the \vocab{trivial $G$-gerbe on $X$} to be $\pi: BG\rightarrow X$, where $BG$ is the category whose objects are pairs $(Y,V)$ with $Y\xrightarrow{f}X$ a scheme over $X$ and $V$ a (left) $G$-torsor on $Y$, and with morphisms $(Y',V')\rightarrow (Y,V)$ given by pairs $(g,\alpha)$ with $g: Y'\rightarrow Y$ a morphism of schemes over $X$ and $\alpha: g_*V'\rightarrow V$ an isomorphism.
\end{ex}

\begin{ex}
Let $G$ be a group scheme on $X$, and let $V$ be a (left) $G$-torsor sheaf.  We associate to $V$ a gerbe $\catx_V$ as follows.  The objects of $\catx_V$ are triples $(Y,W,\alpha)$, where $f:Y\rightarrow X$ is a scheme over $X$, $W$ a (left) $G$-torsor on $Y$, and $\alpha$ an isomorphism of $G$-torsors $\alpha: f_*W\cong V$.
\end{ex}

For reasons that will become apparent momentarily, we denote the set of isomorphism classes of $G$-gerbes on $X$ by $H^2_g(X,G)$.  Note that if $G$ is abelian, then $H^2_g(X,G)$ has a natural group structure
$$H_g^2(X,G)\times H_g^2(X,G)\rightarrow H_g^2(X,G\times G)\rightarrow H_g^2(X,G),$$
induced by the product map on $G$ and the projections $G\rightrightarrows G\times G\rightarrow G$.
\begin{prop}
Suppose that $G$ is an abelian group scheme.  Then there is a canonical isomorphism
$$H^2_g(X,G)\rightarrow H^2(X_{et},G).$$
\end{prop}
\begin{proof}
Suppose that $G'$ a group scheme on $X$, which is injective as a $\bbz$-module.  Let $\catx$ be a $G'$-gerbe, and choose a covering $\mathfrak U=\{f_i: U_i\rightarrow X\}$ on which the gerbe $\catx$ restricts to the trivial gerbe $BG'$.  Then let $f: \amalg_i U_i\rightarrow X$ be the induced covering map.  Then since $G'$ is injective, the natural map $G'\rightarrow f_*f^*G'$ splits.  It follows that $H^2_g(X,G')\rightarrow H^2_g(X,f_*f^*G')$ splits also, and therefore since since the image of $\catx$ is $0$ in $H^2_g(X,f_*f^*G')$, we must have $\catx = 0$.

Now let $G$ be an arbitrary abelian group scheme $G$, and embed $G$ into an injective $G'$, with cokernel $G''$.  Then we have a short exact sequence of abelian group schemes
$$1\rightarrow G\rightarrow G'\rightarrow G''\rightarrow 1,$$
which induces a long exact sequence
$$\dots H^1(X,G')\rightarrow H^1(X,G'')\rightarrow H^2_g(X,G)\rightarrow H^2_g(X,G')\rightarrow H^2(X,G,'').$$
See Milne \cite{milne1980etale} for details.  Then since $H^2_g(X,G') = 0$, this shows that $H^2_g(X,G)$ is the cokernel of $H^1(X,G')\rightarrow H^1(X,G'')$, which is $H^2(X,G)$.  This completes the proof.
\end{proof}

Thus one interpretation of the second cohomology group $H^2(X,G)$ is as isomorphism classes of $G$-gerbes.  Giraud proves that short exact sequences still induce long exact sequences as desired.
\begin{lem}[Giraud]\label{giraurds lemma}
Let $X$ be a scheme and let
$$1\rightarrow G'\rightarrow G\rightarrow G''\rightarrow 1,$$
be a short exact sequence of group schemes on $X$, with $G'\subseteq Z(G)$.  Then there exists a long exact sequence of pointed sets
\begin{align*}
1 & \rightarrow H^0(X,G') \rightarrow H^0(X,G)\rightarrow H^0(X,G'')\\
  & \rightarrow H^1(X,G') \rightarrow H^1(X,G)\rightarrow H^1(X,G'')\xrightarrow{\delta} H^2(X,G')
\end{align*}
where the map $\delta$ in the above sends (the iso. class of) a $G''$-torsor sheaf $V$ on $X$ to the (iso. class of the) associated gerbe $\catx_V$.
\end{lem}
\begin{proof}
See \cite{giraud1971cohomologie}.
\end{proof}

\subsection{Azumaya Algebras and the Brauer Group}
\begin{defn}
Let $X$ be a scheme.  We define an \vocab{Azumaya algebra} $A$ on $X$ to be a twisted form of the $\sheaf O_X$-algebra $M_n$ for some $n$.  The corresponding value of $n$ is called the \vocab{rank}.  We denote the set of isomorphism classes of Azumaya algebras of rank $n$ on $X$ as $\Az_n(X)$.  We call an Azumaya algebra $A$ \vocab{trivial} if $A \cong \END_{\sheaf O_X}(V)$ for some locally free $\sheaf O_X$-module $V$.  We define two Azumaya algebras $A$ and $A'$ to be \vocab{equivalent} if $A\otimes_{\sheaf O_X}(A')^{op}$ is a trivial Azumaya algebra.
\end{defn}

\begin{ex}
If $X = \spec(k)$ for $k$ a field, then an Azumaya algebra on $X$ is the same thing as a central simple $k$-algebra.  Wedderburn's theorem tells us that each central simple algebra is isomorphic to a matrix algebra over a division ring; two central simple algebras are equivalent if and only if they have the same underlying division ring.
\end{ex}

\begin{remk}
With the definition of equivalence provided above, two Azumaya algebras $A$ and $A'$ are equivalent if and only if the corresponding module categories are the equivalent.  In fact, there corresponding equivalence of categories is the one induced by the $A,A'$-bimodule $A\otimes(A')^{op}$.  Thus the definition of equivalence coincides with the notion of Morita equivalence.
\end{remk}

\begin{defn}
We define the (geometric) \vocab{Brauer group} of $X$ to be the set $\Br(X)$ of equivalence classes of Azumaya algebras on $X$.
\end{defn}

\begin{prop}
The Brauer group $\Br(X)$ is an abelian group with binary operation $[A]+[A'] = [A\otimes A']$.
\end{prop}

\begin{ex}
Let $X = \spec(\bbr)$.  Then the central division rings over $\bbr$ consist of only $\bbr$ and the quaternion algebra $Q$.  Thus $\Br(\spec(\bbr)) = \bbz/2$.
\end{ex}

\begin{ex}
Let $X = \bbp^1_k$.  Then $\Br(X) = \Br(\spec(k))$.
\end{ex}

The Brauer group $\Br(X)$ is intimately related to $H^2(X,\bbg_m)$, and in particular there is always an injection $\Br(X)\hookrightarrow H^2(X,\bbg_m)$ mapping $\Br(X)$ into the torsion part of $H^2(X,\bbg_m)$.
\begin{defn}
Let $X$ be a scheme.  We define the \vocab{cohomological Brauer group} $\Br'(X)$ of $X$ to be the torsion subgroup of $H^2(X,\bbg_m)$.
\end{defn}

\begin{remk}
Note that the fppf and etale cohomology of $\bbg_m$ agree.  Hence we simply write $H^i(X,\bbg_m)$ to represent either cohomology group.
\end{remk}

Given an Azumaya algebra $A$ on $X$, we define a $\bbg_m$-gerbe $\catx_A$ by taking the category fibered over $X$ associated to the pseudo-functor
$$(Y\rightarrow X)\mapsto \{(E,\alpha):\text{$E$ loc. free, coh. $\sheaf O_Y$-module, $\alpha: \End_{\sheaf O_Y}(E)\cong A_Y$}\}$$
\begin{lem}
The fibered category $\catx_A$ is a $\bbg$-gerbe, which is trivial if and only if $A$ is a trivial Azumaya algebra.  The map $A\mapsto \catx_A$ defines a group monomorphism of $\Br(X)$ into $H^2_g(X,\bbg)\cong H^2(X,\bbg)$
\end{lem}
\begin{proof}
See Milne \cite{milne1980etale}.
\end{proof}

The image of $\Br(X)$ in $H^2(X,\bbg_m)$, is torsion.  To see this, consider the commutative diagram
$$\xymatrix{
& 1\ar[d] & 1\ar[d]\\
1\ar[r] & \mu_n\ar[d]\ar[r] & \bbg_m\ar[d]\ar[r]^{\cdot^n} & \bbg_m\ar@{=}[d]\ar[r] & 1\\
1\ar[r] & \bbsl_n\ar[d]\ar[r] & \bbgl_n\ar[d]\ar[r]^{\det} & \bbg_m\ar[r] & 1\\
& \bbpgl_n\ar[d] \ar@{=}[r]& \bbpgl_n\ar[d]\\
& 1 & 1
}$$
where the rows and columns are all short exact sequences (in fppf topology).
\begin{prop}
The map $\delta: H^1(X,\bbpgl_n)\rightarrow H^2(X,\bbg_m)$ factors through the map $H^2(X,\mu_n)\rightarrow H^2(X,\bbg_m)$ induced by the Kummer sequence.  In particular, the image of $H^1(X,\bbpgl_n)$ is torsion in $H^2(X,\bbg_m)$.
\end{prop}
\begin{proof}
Follows from mucking about with the above diagram.
\end{proof}
\begin{cor}
The geometric Brauer group embeds into the cohomological Brauer group.
\end{cor}
\begin{proof}
This follows from the fact that $\Br'(X)$ is defined to be the torsion in $H^2(X,\bbg_m)$.
\end{proof}

In ideal situations, one can show that the geometric Brauer group and the cohomological Brauer group are the same.  What exactly constitutes an ``ideal situation" is an open problem.  However, agreement is know to hold for Noetherian schemes with dimension at most $1$.
\begin{prop}
If $X$ is a Noetherian scheme and $\dim(X)\leq 1$, then $\Br(X) = \Br'(X)$.
\end{prop}
\begin{proof}
We reduce to the case that $X$ is reduced.  To do so, we first must compare the sheaves $\sheaf O^*$ and $\sheaf O_{\text{red}}^*$ on $X$.  Define $K_i = 1 + N^i$, for $N$ the nilradical of $X$.  We have a short exact sequence
$$0\rightarrow K_1\rightarrow \sheaf O^*\rightarrow \sheaf O_{\text{red}}^*\rightarrow 0.$$
Moreover, since $X$ is noetherian, there exists an integer $r$ such that $N^r = 0$.  Therefore we have a filtration
$$1 = K_r\subseteq K_{r-1}\subseteq \dots\subseteq K_2\subseteq K_1.$$
Note also that $K_i/K_{i+1}\cong N^i/N^{i+1}$, and therefore the quotients $K_i/K_{i+1}$ are coherent.  Since $N^i/N^{i+1}$ is supported on a $0$-dimensional closed subscheme of $K$, it follows that $H^j(X,K_i/K_{i+1})=0$ for all $j>0$.  Hence the usual fitration trick gives us isomorphisms $H^j(X,K_i)\cong H^j(X,K_{i+1})$ for all $j>0$.  Thus $H^j(X,K_1) = H^j(X,K_r) = 0$ for all $j>0$.  It follows immediately that $H^j(X,\sheaf O^*) = H^j(X,\sheaf O^*_{\text{red}})$ for all $j>0$, and in particular $\Br'(X)=\Br'(X_{\text{red}})$.

Moreover, we have a commutative diagram
$$\xymatrix{
\Br(X)\ar[r]\ar[d] & \Br'(X)\ar[d]^\cong\\
\Br(X_{\text{red}})\ar[r] & \Br'(X_{\text{red}})
}$$
Where the horizontal maps are injective.  Thus if $\Br = \Br'$ for $X_{\text{red}}$, then the above diagram shows that $\Br(X)$ must surject onto $\Br'(X)$, and therefore be an isomorphism.  In this way we can reduce to the case $X = X_{\text{red}}$.

\begin{enumerate}[\text{Case\ }1:]
\item  Assume $\dim(X) = 0$.  Then $X$ is a disjoint union of points, and we can reduce to the case that $X = \spec(K)$.  In this case, equality of $\Br(K)$ and $H^2(\spec(K),\bbg_m)$ is classical result of Galois cohomology.  See \cite{gille2006central} for details.
\item
Assume $\dim(X) = 1$, and let $d\in H^2(X,\bbg_m)$.  We claim that there is a dense, open subscheme $U$ of $X$ on which the restriction of $[d]$ to an element of $H^2(U,\bbg_m)$ lies in the image of $\Br(U)$.  To see this, let $U_0$ be the open subscheme of nonsingular points of $X$.  Then $U_0$ is a disjoint union of nonsingular, integral curves $C_1\cup\dots\cup C_r$.  Let $c_i$ be the generic point of $C_i$ for each $i$.  Since $C_i$ is nonsingular, $\Br'(C_i)\hookrightarrow \Br(K_i)=\Br'(K_i)$, where $K_i$ represents the fraction field of $C_i$.  Thus the restriction $d_i$ of $d$ to $C_i$ restricts to the generic point to a central simple $K_i$-algebra $\Delta_i$.  Let $A_i$ be a maximal $\sheaf O_{C_i}$-order in $\Delta_i$ (discussed in a later section).  Then $(A_i)_{c_i} = \Delta_i$, and therefore there is a Zariski open neighborhood $U_i$ of $c_i$ in $C_i$ on which $A_i$ is Azumaya: take the complement of the zero set of the descriminant.  Since the pullback morphism $\Br'(X)\rightarrow\Br(K_i)$ factors through the injection $0\rightarrow \Br'(U_i)\rightarrow\Br(K_i)$ it follows that the pullback of $d$ to $\Br'(U_i)$ lands in the image of $\Br(U_i)$.  Hence the pullback of $d$ to $\Br'(U)$ for $U=\bigcup_i U_i$ lands in the image of $\Br(U)$.  Since $U$ is dense in $X$, this proves our claim.

So far, we have shown that $d$ is represented on a dense open subscheme $U$ of $X$ by an Azumaya algebra $A$ on $U$.  If we can show that $U$ can be taken to be all of $X$, then we are done!  Let $V$ be a largest dense open subscheme of $X$ on which $d$ is represented by a sheaf of Azumaya algebras (this makes sense since the complement of $U$ is finitely many points).  Suppose $x\in X\diff V$.  By the previous paragraph, we know that $x$ is a closed point of $X$, and therefore $R = \sheaf O_{X,x}$ is a one-dimensional local ring.  Set $V_x = \spec(R)\cap V = \spec(R)\diff\{x\}$.  The completion $\wh R$ of $R$ is a Hensilian local ring, and therefore $\Br(\wh R) = \Br'(\wh R)$ \cite{milne1980etale}.  Hence the pullback of $d$ to $\spec(\wh R)$ may be represented by an Azumaya $\wh R$-algebra $\wh B$, which must be equivalent to $A$ on $\wh V_x = \spec(\wh R)\diff \{x\}$.  Hence there exist integers $s,t$ such that $M_s(A) = M_t(\wh B)$ on $\wh V_x$.  Then there exists a unique Azumaya $R$-algebra $B$ such that $B\otimes\wh R \cong M_t(\wh B)$.  Gluing $B$ and $A$ together on $\{x\}\cup V$, extends $A$ to an Azumaya algebra on $V\cup \{x\}$.  This contradicts the maximality of $V$, and hence $V = X$.  This completes the proof.
\end{enumerate}
\end{proof}

Furthermore, if $X$ is a nonsingular, integral surface then $\Br(X) = \Br(X')$.  We will prove this result in a later section.


