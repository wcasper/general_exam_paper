\section{Classical Theory of Maximal Orders}
\subsection{Basic Definitions and Facts}
Let $R$ be a noetherian integral domain with fraction field $K$, and let $\Delta$ be a central simple $K$-algebra, ie. a simple $K$-algebra with center $K$.
\begin{defn}
An \vocab{$R$-order} in $\Delta$ is an $R$-subalgebra $A$ of $\Delta$, finitely generated as an $R$-module, such that $A\otimes_R K = \Delta$.  The set $\Ord_{R,\Delta}$ forms a poset with ordering defined by inclusion.  A \vocab{$R$-order in $\Delta$} is \vocab{maximal} if it is not properly contained in any $R$-order in $\Delta$
\end{defn}
Note that it follows from the definition that if $A$ is an $R$-order in $\Delta$, then the center of $A$ contains $R$.
\begin{lem}
Let $S\subseteq R\diff\{0\}$ be a multiplicative subset of $R$.  Then there is a morphism of posets
$$S^{-1}: \Ord_{R,\Delta}\rightarrow \Ord_{S^{-1}R,\Delta},\ \ A\mapsto S^{-1}A$$
which is surjective and sends maximal orders to maximal orders.
\end{lem}
\begin{proof}
If $A$ is an $R$-order in $\Delta$, then the fact that $S^{-1}A$ is an $S^{-1}R$-order in $\Delta$ follows from the exactness of localization.

Now suppose that $A'$ is an $S^{-1}R$-order in $A$, and let $a_1,\dots, a_m$ be a set of generators for $A'$ as an $S^{-1}R$-module.  Set $M = Ra_1 + \dots + Ra_m$ and let
$$A = \{a\in A': Ma\subseteq M\}.$$
It is clear from the definition that $A$ is an $R$-subalgebra of $A'$ and that $S^{-1}A = A'$; in particular this also tells us that $A\otimes_R K = \Delta$.  Thus to show that $A$ is an $R$-order in $\Delta$, all that is left to show is that $A$ is finitely generated as an $R$-module.  To see this, note that the action of $A$ on $M$ on the right induces an $R$-module monomorphism of $A$ into the finitely generated $R$-module $\End_R(A)$.  Since $R$ is noetherian, it follows that $A$ is finitely generated.

Lastly, suppose that $A$ is a maximal $R$-order in $\Delta$, and let $A' = S^{-1}A$, and suppose that $A'$ is contained in another $S^{-1}R$-order $B'$.  Then there exists a nonzero $r\in R$ such that $rB'\subseteq A'$.  Define $I = A\cap rB'$, then $I$ is a two-sided ideal of $A$ with $S^{-1}I=rB'$.  Define $B = \{b\in B': Ib\subseteq I\}$.  By a similar argument to the one of the previous paragraph, $B$ is an $R$-order in $\Delta$ and $S^{-1}B = B'$.  Furthermore, $B$ contains $A$ and therefore by maximality $B = A$.  Hence $B'=S^{-1}B = S^{-1}A = A'$.
\end{proof}

\begin{cor}
Let $A$ be an $R$-order in $\Delta$.  Then the following are equivalent.
\begin{enumerate}[(a)]
\item  $A$ is maximal
\item  $A_\p$ is maximal for every prime ideal $\p$ of $R$
\end{enumerate}
If in addition $R$ is integrally closed in its fraction field, then (a) or (b) is also equivalent to
\begin{enumerate}[(a)]
\setcounter{enumi}{2}
\item  $A$ is a reflexive $R$-module and $A_\p$ is maximal for all $\p\in\spec(R)$ with $\height(\p)=1$.
\end{enumerate}
\end{cor}
\begin{proof}
The fact that (a) implies (b) follows immediately from the previous lemma.  Conversely, assume (b) and suppose that $B$ is any $R$-order containing $A$.  Then $A_\p\subseteq B_\p$ for all prime ideals $\p$, and therefore by maximality, $A_\p = B_\p$ for all prime ideals $\p$.  It follows that $A = B$, and hence $A$ is maximal.

Next suppose that $R$ is also integrally closed.  Note that $A^{**}$ is an $R$-order containing $A$, and therefore if $A$ is maximal $A$ is reflexive.  Hence by the previous lemma (a) implies (c).  Conversely, assume (c).  Suppose that $B$ is any $R$-order containing $A$.  Then by maximality at height one primes, we have that $A_\p = B_\p$ for all $\p$ with $\height\p = 1$.  Then since $R$ is integrally closed,
$$A^{**} = \bigcap_{\height\p = 1} A_\p = \bigcap_{\height\p = 1} B_\p = B^{**}.$$
Since $A$ is reflexive, this shows that $A = B^{**}$.  However, $B^{**}$ is an $R$-order containing $B$, and therefore $A$ contains $B$.  Hence $A = B$, and therefore $A$ is maximal.  Thus (c) implies (a).
\end{proof}

\begin{defn}
Let $A$ be an $R$-order in $\Delta$.  We define the \vocab{reduced norm} $\nrd(a)$, \vocab{reduced trace} $\trd(a)$, and \vocab{reduced characteristic polynomial} $\chrd(x;a)$ to be the corresponding data as elements of the Azumaya algebra $\Delta$.
\end{defn}

\begin{lem}
Suppose that $R$ is integrally closed and $A$ is an $R$-order in $\Delta$.  Then for all $a\in A$, the reduced characterisic polynomial $\chrd(x;a)$ is in $R[x]$.  Consequently, the reduced norm and reduced trace are also both in $R$.
\end{lem}
\begin{proof}
Let $a\in A$.  Then since $A$ is finitely generated over $R$, $a$ is integral over $R$.  Let $f(x)\in K[x]$ be the minimal polynomial of $a$; since $R$ is integrally closed, $f(x)\in R[x]$.  Note that $f(x)$ and $\chrd(x;a)$ have the same roots over the algebraic closure $\ol K$ of $K$ (not counting algebric multiplicity), since they are exactly the eigenvalues of the left action of $a\otimes 1$ on $\Delta\otimes_K\ol K$.  This means that the roots of $\chrd(x;a)$ are all algebraic over $R$.  Hence the coefficients of $\chrd(x;a)$ are algebraic over $R$.  Since $\chrd(x;a)\in K[x]$ and $R$ is algebraically closed in $K$, this implies that $\chrd(x;a)\in R[x]$.
\end{proof}

\begin{prop}
Let $R$ be integrally closed.  Then every $R$-order $A$ in $\Delta$ is contained in a maximal $R$-order in $\Delta$.
\end{prop}
\begin{proof}
Let $B$ be any $R$-order in $\Delta$.  Then the reduced trace defines a nondegenerate bilinear form on $B$.  Therefore the map $B\mapsto B^*$ defined by
$$b\mapsto b^*: x\mapsto \trd(xb)$$
is an injective $R$-module homomorphism.

Let $\Omega\subseteq\Ord_{R,\Delta}$ be the set of all $R$-orders in $\Delta$ containing $A$.  Then $\Omega$ is nonzero, since it contains $A$, moreover if $B_1\subseteq B_2\subseteq \dots$ is a chain in $\Omega$, then $B:=\bigcup_i B_i$ is an $R$-subalgebra of $A$.  Moreover the $R$-module monomorphisms $B_i\rightarrow B_i^*\subseteq A^*$ induce an $R$-module monomorphism $B\rightarrow A^*$.  Since $A^*$ is a finitely generated $R$-module and $R$ is noetherian, it follows that $B$ is finitely generated.  Hence $B\in \Omega$.  Thus Zorn's lemma tells us that $A$ is contained in a maximal $R$-order in $\Delta$.
\end{proof}

\begin{prop}
If $A$ is an Azumaya algebra on $R$, then $A$ is a maximal $R$-order in $\Delta$.
\end{prop}
\begin{proof}
We claim that the natural map $A\mapsto A^*$ given by $a\mapsto a^*: x\mapsto \trd(xa)$ is an isomorphism.  Note that by the nondegeneracy of the reduced trace on $\Delta$, this map is injective, so it suffices to show that it is surjective.  To see this, choose a basis $a_1,\dots, a_N$ of $A$ as a free $R$-module and consider the matrix $T = \trd(a_ia_j)\in M_N(R)$.  Since $A$ is an Azumaya algebra, the discriminant $d = \det(T)$ of $A$ is a unit.  Therefore $T$ is invertible in $R$.  Hence for $\vec r = T^{-1}\vec e_1$, and $a = \sum_i a_ir_i$, we have
$$\trd(a_1a) = \vec e_1\cdot T\vec r = 1.$$
Thus $A$ contains an element with reduced trace $1$, namely $y$.  For any $\varphi\in\Hom_R(A,R)$, define $\psi\in\End_R(A)$ by $\psi(x) = \varphi(x)y$.  Then since $A$ is Azumaya, the natural map $A\otimes_RA^{op}\rightarrow\End_R(A)$ is an isomorphism, and therefore there exists $u,v\in A$ such that $\psi(x) = uxv$.  It follows that
\begin{align*}
\varphi(x)
  & = \varphi(x)\trd(y) = \trd(\varphi(x)y)\\
  & = \trd(\psi(x)) = \trd(uxv) = \trd(xvu) = (vu)^*(x).
\end{align*}
This proves that $A\mapsto A^*$ is an isomorphism.  Now if $B$ is any $R$-order in $\Delta$ containing $A$, then $A\subseteq B\rightarrow B^*\subseteq A^*$, from which it follows that $A = B$.
\end{proof}

\begin{ex}
Let $R = k[x]$, and let $\Delta = M_N(k(x))$.  Then
$$A = \left(\begin{array}{cc}
 R & R\\
xR & R
\end{array}\right)$$
is an $R$-order in $\Delta$.  It is not maximal, since it is contained in the $R$-order $M_N(R)$.  Furthermore, $M_N(R)$ is an Azumaya algebra, and therefore is a maximal $R$-order in $\Delta$ containing $A$.  Note that it is not the only $R$-order in $\Delta$ containing $A$, since for example $A$ is contained in the $R$-order
$$B = \left(\begin{array}{cc}
 R & x^{-1}R\\
xR & R
\end{array}\right).$$
Note that $A$ and $B$ are isomorphic under an inner automorphism of $\Delta$, and therefore $B$ is also a maximal $R$-order in $\Delta$.
\end{ex}

Note that in the previous example, both of the maximal $R$-orders in $\Delta$ are Azumaya algebras -- and in fact all of the maximal $R$-orders in $\Delta$ are Azumaya.  This is indicative of the following more general phenomenon, which we discuss in greater detail in the section below on ramification.

\subsection{Maximal Orders over a complete DVR}
Let $R$ be a complete DVR with valuation $\nu$, uniformizer $\pi$, residue field $k$, and fraction field $K$.  Let $\Delta$ be a central simple $K$-algebra; by Wedderburn's Theorem, we may take $\Delta = M_N(D)$, for $D$ a division algebra with center $K$, with $\dim_K(D) = m$.  In this section, we will show that the maximal $R$-order in $\Delta$ is essentially unique.  More specifically, we will show that there exists a unique maximal $R$-order $A$ in $D$, and that any maximal $R$-order in $\Delta$ is isomorphic to $M_N(A)$ via an inner automorphism of $\Delta$.  The exposition we provide is based on that found in Reiner \cite{reiner1975maximal}, which in turn is based an earlier paper of Hasse.

The main ideal is that the function $w: D\rightarrow K$ defined by
$$w(x) = \frac{1}{m}\nu(\det(x)),\ \ \forall x\in D$$
extends the valuation $\nu$ to a valuation on all of $D$.  Our next goal is to prove this, and it will take a bit of doing.

\begin{lem}
Suppose $f\in K[x]$ is an irreducible polynomial with
$$f(x) = a_nx^n + a_{n-1}x^{n-1} + a_1x + a_0.$$
Then for all $0\leq i\leq n$, we have that
$$\nu(a_i)\geq \min\{\nu(a_0),\nu(a_n)\}.$$
\end{lem}
\begin{proof}
We will assume that there exists $i$ such that $\nu(a_i) <\min\{\nu(a_0),\nu(a_n)\}$ and will arrive at a contradiction.  Let $t = \min_i \nu(a_i)$, and set
$$m = \min\{0\leq i\leq n: \nu(a_i) = t\}.$$
Note that $a_m\neq 0$.  Then since $f$ is irreducible in $K[x]$, so too is the polynomial
$$g(x) := a_m^{-1}f(x) = b_nx^n + b_{n-1}x^{n-1} + \dots + b_0,$$
for $b_i = a_i/a_m$.  Note that by definition $m\geq 1$ and for all $0\leq i<m$, we have $b_i\in \pi R$.  Therefore reducing $g$ modulo the maximal ideal of $R$, we find
$$\ol g(x) = x^m(\ol b_nx^{n-m} + \dots + \ol b_{m+1}x + \ol b_{m}).$$
By Hensel's lemma, it follows that $g(x)$ is reducible, a contradiction.  This proves our lemma.
\end{proof}

\begin{lem}
Let $a\in D$ and let $f(x)\in K[x]$ be it's minimal polynomial.  Then
$$w(a) = \frac{1}{[K(a):K]} \nu(f(0)).$$
Consequently, $w(a)\geq 0$ if and only if $a$ is integral over $R$.
\end{lem}
\begin{proof}
Let $g(x)$ be the characteristic polynomial of $a$, viewed as an endomorphism of $D$, and let
$$g(x) = x^m + \dots + (-1)^m\det(a).$$
and must divide a power of $f(x)$.  Hence $g(x) = f(x)^{m/r}$, from which it follows that
$$(-1)^m\det(a) = f(0)^{m/r},$$
and therefore
$$w(a) = \frac{1}{m}\nu(\det(a)) = \frac{1}{r}\nu(f(0)) = \frac{1}{[K(a):K]}\nu(f(0)).$$

If $a$ is integral over $R$, then $g(x)\in R[x]$, and therefore $\det(a)\in R$.  Hence $w(a)\geq 0$.   Conversely, suppose that $w(a)\geq0$, then $\nu(f(0))>0$ and therefore by the previous lemma $f(x)\in R[x]$.  Consequently, $a$ is integral over $R$.
\end{proof}

We are now in a position to show that $w$ defines a discrete valuation on $D$.
\begin{thm}
Let $D$ be a division $K$-algebra with $\dim_K(D) = m<\infty$.  Then the valuation $\nu$ extends to a discrete valuation $w$ on $D$, defined by
$$w(x) = \frac{1}{m}\nu(\det(x)),\ \forall x\in D$$
where $\det(x)$ is the determinant of $x$ viewed as an endomorphism of $D$ by left multiplication.
\end{thm}
\begin{proof}
It's clear from the definition that $w|_K = \nu$.  To prove that $w$ is a discrete valuation on $D$, we must verify the following
\begin{enumerate}[(a)]
\item  $w(x) = \infty$ if and only if $x = 0$
\item  $w(ab) = w(a) + w(b)$ for all $a,b$
\item  $w(D)\cong \bbz$
\item  $w(a+b)\geq \min\{w(a),w(b)\}$
\end{enumerate}
Note that (a), (b) follow immediately from the definition and the fact that $D$ is a division algebra.  Furthermore, $w(D)$ is an infinite subgroup of $\bbz$, and therefore (c) holds.  Thus it suffices to prove (d).

Suppose that $w(a)\geq 0$, and let $f(x)$ be the minimal polynomial of $a$.  Then by the previous lemma $f(x)\in R[x]$, and since $f(x-1)$ is the minimal polynomial of $a+1$, it follows that $a+1$ is integral over $R$.  Hence by the previous lemma, $\nu(a +1)\geq 0$.  Thus
$$\nu(a+1) \geq 0 = \min\{\nu(a),\nu(1)\}.$$
Now take $b,c\in K$ and without loss of generality assume $\nu(b)\leq \nu(c)$.  Then
$$\nu(cb^{-1} + 1)\geq \min\{\nu(c/b),\nu(1)\}$$
and by adding $\nu(b)$ to both sides, we obtain (d).
\end{proof}

As it turns out, the way that we defined $w$ is in fact the only way that one may extend the valuation $\nu$ to $D$.  To prove this, we establish the following lemma.
\begin{lem}
Suppose $f\in K[x]$ is an irreducible monic polynomial, with
$$f(x) = x^n + a_{n-1}x^{n-1} + \dots + xa_1 + a_0.$$
Then for all $0\leq i\leq n-1$,
$$\nu(a_i)\geq \frac{n-i}{n}\nu(a_0).$$
\end{lem}
\begin{proof}
Let $L$ be the splitting field of $f$, and let $\nu'$ be the extension of $\nu$ to $L$ given by $\nu'(x) = \frac{1}{n}\nu(\det(x))$.  Let $b_1,\dots,b_n\in L$ be the roots of $f$.  Then for all $i$, $f(x)$ is the characteristic polynomial of $b_i$; hence $a_0 = f(0) = (-1)^n\det(b_i)$, and therefore $\nu'(b_i) = \frac{1}{n}\nu(a_0)$.  Moreover $f(x) = \prod_i(x-b_i)$, making the coefficients of $f$ the various elementary symmetric polynomials in $b_i$ and therefore
$$\nu(a_i) = \nu'(a_i) \geq \frac{n-i}{n}\nu(a_0).$$
\end{proof}

\begin{cor}
The valuation $w$ is the unique extension of $\nu$ to $D$.
\end{cor}
\begin{proof}
Suppose that $w'$ is another extension of $\nu$ to $D$ that doesn't agree with $w$.  Then there exists $a\in D$ with $w'(a)>w(a)$.  Let
$$f(x) = a_nx^n + a_{n-1}x^{n-1} + \dots + a_0$$
be the minimal polynomial of $f(x)$ (note that $a_n = 1$).  Then we have that
$$a_0 = -\sum_{i=1}^{n}a_ia^i,$$
and therefore
$$\nu(a_0) = w'(a_0) \geq \min_i\{\nu(a_i) + iw'(a)\} > \min_i\{\nu(a_i) + iw(a)\}.$$
Furthermore, by the previous lemma
$$\nu(a_i)\geq \frac{n-i}{n}\nu(a_0) = (n-i)w(a),$$
and therefore the above inequality tells us
$$\nu(a_0) > nw(a),$$
which is a contradiction.  Thus $w = w'$.
\end{proof}

Using the valuation $w$, it is now easy to explicitly define the maximal $R$-order in $D$.  It is determined by
$$A = \{a\in D: w(a)\geq 0\}.$$
It's clear from the definition that $A$ is an $R$-subalgebra of $D$, and that $A\otimes_RK = D$.  Furthermore, if $B$ is any other $R$-order in $D$, then every $b\in B$ is integral over $R$.  This implies that $w(b)\geq 0$, and therefore that $b\in A$.  Thus $A$ containes all other $R$-orders in $D$.  Thus to show that $A$ is a maximal $R$-order in $D$, it suffices to prove that $A$ is a finitely generated $R$-module.  To prove this, we need to more finely probe the structure of $A$.

We fix an element $\pi_D\in A$ satisfying $\nu_D(\pi_D) = 1$.
\begin{prop}
Let $A$ be the $R$-algebra defined above.  Then
\begin{enumerate}[(a)]
\item $J(A) = \pi_DA$ is the unique maximal left (or right or two-sided) ideal of $A$
\item $J(A)\cap R = \pi R$
\item $A/J(A)$ is a division $k$-algebra
\item if $I$ is a nonzero left (or right ideal of $A$, then $I = J(A)^m = \pi_D^mA$ for some $m$
\end{enumerate}
\end{prop}
\begin{proof}
Each of these statements follows from mucking about with $v_D$.
\end{proof}

\begin{defn}
Let $A$ be a maximal $R$-order in $D$.  We define the \vocab{ramification index} $e = e(D/R)$ of $D$ over $K$ to be the unique integer $e$ satisfying $J(A)^e = \pi A$.  We define the \vocab{inertial degree} $f = f(D/R)$ to be $f = \dim_k(A/\pi A)$.
\end{defn}

\begin{thm}
The $R$-algebra $A$ is a finitely generated free $R$-module of rank $n = \dim_K(D)$.  Moreover, if $e$ and $f$ are the ramification index and inertial degree, respectively, then $ef = n$.
\end{thm}
\begin{proof}
Let $e$ be the ramification index.  Choose $a_1,\dots, a_f\in A$ with $\ol a_1,\dots \ol a_f\in A/\pi A$ a $k$-basis.  We claim that the set
$$\{a_i\pi^j: 1\leq i\leq f,\ 0\leq j\leq e-1\},$$
is $K$-linearly independent.  To see this, suppose otherwise.  Then there exist $\alpha_{ij}\in K$ with
$$\sum_{ij}\alpha_{ij}a_i\pi^j = 0.$$
For each $j$ set $n_j = \min_i\nu(\alpha_{ij})$, and define $\beta_{ij}$ by $\alpha_{ij} = \pi^{n_j}\beta_{ij}$.  Then $\beta_{ij}\in R$ for all $i,j$, and for all $i$ there exists $j$ with $\beta_{ij}\notin\pi R$.  Define $b_j = \pi^{n_j}\pi_D^j$ and $\sigma$ be a permutation of $\{0,\dots, e-1\}$ such that $\nu_D(b_{\sigma(0)}) < \dots <\nu_D(b_{\sigma(e-1)})$, set $c_j = b_{\sigma(j)}/b_{\sigma(0)}$ and $\gamma_{ij} = \beta_{i\sigma^{-1}(j)}$.  Then $\sum_{ij}\gamma_{ij}a_ib_j=0$, with $\gamma_{ij}\in R$ for all $i,j$, $\gamma_{i1}\notin\pi R$ for some $i$ and $c_j\in \pi A$ for all $j\geq 2$.  Therefore reducing modulo $\pi A$ we obtain $\sum_{i}\ol \gamma_{i1}\ol a_i = 0$, which contradicts the assumption of $k$-linear independence.

Lastly, we claim that $A = R\{a_i\pi_D^j\}$.  To see this, suppose that $x\in A$, and let $\nu_D(x) = ke + j$ for integer $j,k$ with $0\leq j < e$.  Then we may write $x = \pi^k\pi_D^ju$ for some $u\in A^\times$.  Therefore by definition of the $a_i$'s, there exist $r_i\in R$ such that $u = \sum_i r_ia_i$ modulo $\pi A$.  Setting $x_1 = \pi^k\pi_D^j(u-\sum_i r_ia_i)$ we have that $x = y_1 + x_1$ for some $y_1\in R\{a_i\pi_D^j\}$ and $x_1\in A$ with $\nu_D(x_1)>\nu_D(x)$.  Repeating this process, we obtain a sequence $x_i = y_{i+1}+x_{i+1}$ with $\nu_D(x_{i+1})>\nu_D(x_i)$ and for all integers $m>0$
$$x = \sum_{\ell=1}^m y_\ell + x_m = \sum_{i,j}r_{ijm}a_i\pi_D^j + x_m.$$
The for each $i,j$ the sequence of coefficients $r_{ijm}$ is Cauchy sequences in $R$, and since $R$ is complete, converge to a limit $r_{ij}$.  It follows that $x = \sum_{i,j}r_{ij}a_i\pi_D^j$, proving our claim.
\end{proof}

\begin{cor}
The algebra $A$ is the unique maximal $R$-order in $D$.
\end{cor}
\begin{proof}
Note that $A$ is the integral closure of $R$ in $\Delta$.  The previous theorem shows that $A$ is finitely generated as an $R$-module, and it is clear from the definition that $A$ is an $R$-subalgebra of $\Delta$ with $A\otimes_RK=\Delta$.  Therefore $A$ is an $R$-order in $\Delta$, and since every $R$-order in $\Delta$ consists of elements integral over $R$, it is necessarily maximal.
\end{proof}

\begin{thm}
The matrix algebra $M_N(A)$ is a maximal $R$-order in $\Delta$.  Furthermore
\begin{enumerate}[(a)]
\item  $J(A) = \pi_DM_N(A)$ is the unique maximal two-sided ideal of $A$
\item  if $I$ is any two-sided ideal of $A$, then $I = J(A)^n = \pi_D^n M_N(A)$ for some $n$
\item  if $B$ is any other maximal $R$-order in $A$, then $B$ is of the form $u M_N(A) u^{-1}$ for some invertible $u\in\Delta$
\end{enumerate}
\end{thm}

\subsection{Maximal orders over DVRs}
In this section, let $R$ be an arbitrary DVR with uniformizer $\pi$ and fraction field $K$.  In the last section, we determined a great many structural theorems regarding maximal orders over complete DVRs.  In this section, we will show how the theory of maximal orders over $R$ may be related to the theory of maximal orders over their completion $\wh R$.  For notational convenience, in this section $\Delta$ will be a CSA/$K$, $\wh K$ will be the fraction field of $\wh R$, and $\wh\Delta$ will be $\Delta\otimes_K \wh K$.  In particular, $\wh \Delta$ is a CSA/$\wh K$, and $\Delta$ is a division algebra if and only if $\wh \Delta$ is a division algebra.

The basic idea of relating the theory of maximal orders on $R$ to the theory of maximal orders on $\wh R$ is based on fpqc descent.  In particular, take $X = \spec(R)$, $U_1 = \spec(\wh R)$ and $U_2 = \spec(K)$.  Then the two natural maps $f_i: U_i\rightarrow X$ form an fppf covering of $X$.  Descent data on this cover consists of a sheaves $F_i$ on $U_i$ with isomorphisms $\varphi_{ij}: p_{ij1}^*F_i\rightarrow p_{ij2}^*F_j$ for all $1\leq i\leq j\leq 2$, where $p_{ijk}:U_i\times_X U_j\rightarrow U_k$ is the natural projection.  In our case, we have canonical identifications $U_1\times_X U_1 = U_1$, $U_2\times_X U_2 = U_2$ and $U_1\times_X U_2 = \spec(\wh K)$ with $p_{iij} = \id_{U_i}$ and $p_{12j}:\spec(\wh K)\rightarrow U_j$ the natural map.  Therefore the only nontrivial descent data comes from the isomorphisms $\varphi_{ij}$ for $i\neq j$.  Unwinding definitions, descent theory then tells us that
$$M\mapsto (M\otimes_R\wh{R},M\otimes_R K)$$
defines a bijective correspondence between $R$-modules $M$ and pairs $(\wh M, V)$ with $\wh M$ an $\wh R$-module and $V$ a $K$-vector space satisfying $\wh M\otimes_{\wh R} \wh K = V\otimes_K \wh K$.

\begin{prop}
Let $\wh\Delta = \Delta\otimes_K\wh K$.  Then $\wh\Delta$ is a central simple $\wh{K}$-algebra and there is a bijective morphism of posets
$$\Ord_{R,\Delta}\rightarrow\Ord_{\wh{R},\wh\Delta},\ \ A\mapsto A\otimes_R\wh{R}.$$
In particular $A$ is a maximal $R$-order in $\Delta$ if and only if $A\otimes_R\wh{R}$ is a maximal $\wh{R}$-order in $\wh{\Delta}$.
\end{prop}
\begin{proof}
Taking $V = \Delta$ in the above fpqc correspondence shows that $M\mapsto M\otimes_R\wh R$ is a bijective correspondence between $R$-modules $M$ satisfying $M\otimes_R K = \Delta$ and $\wh R$-modules $\wh M$ satisfying $\wh M\otimes_{\wh R}\wh K = \wh \Delta$.  This restricts to the desired poset isomorphism.
\end{proof}
Thus the study of maximal orders on $R$ is closely tied to the study of maximal $\wh R$-orders on $\wh \Delta$.  Using this, we deduce some of the structure of maximal $R$-orders in $\Delta$.

\begin{cor}
If $A$ is a maximal $R$-order in $\Delta$, then $A$ is left and right hereditary.
\end{cor}

\begin{prop}
Let $A$ be an $R$-order in $\Delta$, and let $\ol A = A/\pi A$.  Then
$$A/J(A)\cong \ol A/J(\ol A)\cong (\wh A)/J(\wh A).$$
\end{prop}
\begin{proof}
By elementary ring theory, $A/J(A)\cong \ol A/J$ and $\wh A/J(\wh A)\cong \ol{\wh A}/J(\ol{\wh A})$, where $\ol{\wh A} = \wh A/\pi\wh A$.  Therefore it suffices to show that $\ol A\cong\ol{\wh A}$.  This in turn follows from the fact that $\pi\wh A \cong \pi A \otimes\wh R$ and the exactness of $-\otimes\wh R$.
\end{proof}

\begin{thm}
Let $A$ be a maximal $R$-order in $\Delta$.  The ideal $J(A)$ is the unique maximal left (or right) ideal of $A$, and furthermore
$$J(\wh A) = J(A)\otimes\wh R,\ \ J(\wh A)\cap A = J(A).$$
In fact, every nontrivial two-sided ideal of $A$ is a power of $J(A)$.
\end{thm}
\begin{proof}
We have a natural bijective correspondence
$$\{\text{maximal left ideals of $A$}\}\longleftrightarrow \{\text{maximal left ideals of $A/J(A)$}\}$$
and
$$\{\text{maximal left ideals of $\wh A$}\}\longleftrightarrow \{\text{maximal left ideals of $\wh A/J(\wh A)$}\}$$
so by the previous proposition, there is a bijective correspondence between maximal left ideals of $A$ and maximal left ideals of $\wh A$.  Since $\wh A$ has exactly one maximal left ideal, it follows that $A$ has exactly one also.  Then, since $J(A)$ is the intersection of all maximal left ideals, $J(A)$ must be the maximal left ideal of $A$.  Furthermore, tracing the above correspondence, we see that $J(A)\otimes\wh R = J(\wh A)$ and $J(A) = J(\wh A)\cap A$.

Now if $I$ is a nontrivial two-sided ideal of $A$, then $I\otimes K$ is a nontrivial two-sided ideal of $\Delta$.  Hence $I\otimes K = \Delta$, and therefore there exists $a_i\in I$ and $r_i\in K$ with $\sum_i a_ir_i = 1$.  Choose $r\in R$ such that $rr_i\in R$ for all $i$, and set $s = \nu(r)$.  Then $r = \sum_i a_i(rr_i)\in I$ and therefore $rA = \pi^sA\subseteq I$ for all $I$.  Therefore $I$ is the preimage of a two-sided ideal of $A/\pi^sA\cong \wh A/\pi^s\wh A$.  Therefore $I\otimes \wh R$ is a preimage of a two-sided ideal of $\wh A/\pi^s\wh A$, and therefore $I\otimes\wh R = \pi^t \wh A = J(\wh A)^t$ for some $t\leq s$.  This tells us that $I = (I\otimes \wh R)\cap A = J(\wh A)^t\cap A = J(A)^t$.
\end{proof}

\begin{lem}
Let $A$ be a maximal $R$-order in $\Delta$, and let $M$ be a finitely generated left $A$-module.  Then
\begin{enumerate}[(a)]
\item  $A$ is a projective $A$-module
\item  $A$ is a free $R$-module
\item  if $B$ is another finitely generated free $R$-module, then $A\cong B$ as $A$-modules if and only if $A$ and $B$ have the same rank as $R$-modules
\end{enumerate}
\end{lem}

\begin{thm}
Let $A$ be a maximal $R$-order in $\Delta$.  Then
\begin{enumerate}[(a)]
\item  Every one-sided ideal of $A$ is principal.
\item  If $B$ is another maximal $R$-order in $\Delta$ then $B = uAu^{-1}$ for some $u\in \Delta^\times$.
\item  Let $\wh\Delta = M_t(\wh D)$ for $\wh D$ a division algebra over $\wh K$, and let $\wh C$ be a unique maximal $R$-order in $\wh D$.  Then $\wh C/J(\wh C)$ is a division algebra over $k$ and $A/J(A) = M_t(\wh C/J(\wh C))$.
\end{enumerate}
\end{thm}

\subsection{Normal Orders}
Let $R$ be a DVR with fraction field $K$, maximal ideal $\m$, uniformizer $\pi$, and residue field $k$.  Also let $\Delta = M_N(D)$ with $D$ a CDA over $K$.
\begin{defn}
Let $A$ be an $R$-order in $\Delta$.  Then $A$ is called \vocab{normal} if $J(A) = At$ for some $t\in A$.
\end{defn}
Note in particular that maximal $R$-orders are normal.
\begin{lem}
If $A$ is a normal $R$-order in $\Delta$, then $A$ is hereditary.
\end{lem}
\begin{proof}
See \cite{Artin&deJong}.
\end{proof}
\begin{lem}
Let $R'$ be a DVR with fraction field $K'$, let $\Delta' = \Delta\otimes_KK'$ and suppose that $R\rightarrow R'$ is an etale morphism of rings.  Then if $A$ is a normal $R$-order in $\Delta$, $A\otimes_R R'$ is a normal $R'$-order in $\Delta'$.  Furthermore, if $J(A) = At$, then $J(A') = A'(t\otimes 1)$.
\end{lem}
\begin{proof}
See \cite{Artin&deJong}.
\end{proof}
In the case that $R$ is complete and $k$ is sufficiently nice, the normal $R$-orders in $\Delta$ are known up to isomorphism.  In particular, if $B$ is the unique $R$-order in $D$, then the isomorphism class of a normal $R$-order in $\Delta$ is determined by its ramification index and inertia degree.
\begin{defn}
Let $e,f$ be positive integers.  We define the \vocab{standard hereditary order} $A_{e,f}(B)$ by
$$A_{e,f}(B)= A_{e,1}(B)\otimes_R M_f(R),$$ with
$$A_{e,1}(B)= \left(\begin{array}{ccccc}
B      &  B     & \dots  & B      & B\\
J(B)   &  B     & \dots  & B      & B\\
\vdots & \vdots & \ddots & \vdots & \vdots\\
J(B)   & J(B)   & \dots  & B      & B\\
J(B)   & J(B)   & \dots  & J(B)   & B
\end{array}\right)\subseteq M_e(B).$$
\end{defn}

\begin{lem}\mbox{}
Let $R$ be a complete DVR, and $ef=N$.  Also let $B$ be the unique maximal $R$-order in $D$, with $J(B) = \pi_DB =B\pi_D$.  Then
\begin{enumerate}[(a)]
\item  $A_{e,f}(B)$ is an $R$-order in $\Delta$
\item  $J(A_{e,f}) = J(A_{e,1})\otimes_R M_{f}(R),$ with
$$A_{e,1}(B)= \left(\begin{array}{ccccc}
J(B)   &  B     & \dots  & B      & B\\
J(B)   &  B     & \dots  & B      & B\\
\vdots & \vdots & \ddots & \vdots & \vdots\\
J(B)   & J(B)   & \dots  & J(B)   & B\\
J(B)   & J(B)   & \dots  & J(B)   & J(B)
\end{array}\right)\subseteq M_e(B).$$
\item  $A_{e,f}$ is normal, and in fact $J(A_{e,f}) = A_{e,f}t$ for
$$t = \left(\begin{array}{cccccc}
0      &  1     & 0      & \dots  &  0     & 0\\
0      &  0     & 1      & \dots  &  0     & 0\\
\vdots & \vdots & \vdots & \ddots & \vdots & \vdots\\
0      &  0     & 0      & \dots  &  0     & 1\\
\pi    &  0     & 0      & \dots  &  0     & 0
\end{array}\right)\otimes_R I.$$
\end{enumerate}
\end{lem}
\begin{proof}
See \cite{Artin&deJong}.
\end{proof}

\begin{thm}
Let $R$ be a complete DVR, and let $B$ be the unique maximal $R$-order in $D$.  Then an $R$-order in $\Delta = M_N(D)$ is normal if and only if it is isomorphic to $A_{e,f}(B)$ for some integers $e,f$.
\end{thm}
\begin{proof}
See \cite{Artin&deJong}.
\end{proof}

\subsection{Ramification of Maximal Orders over DVRs}
In this section $R$ will always be a DVR with fraction field $K$, uniformizer $\pi$ and residue field $k$, and $\Delta$ will represent a central simple $K$-algebra of rank $n$.
\begin{defn}
Let $A$ be a maximal $R$-order in $\Delta$.  The central simple algebra $\Delta$ is said to \vocab{ramify} over $R$ if $A$ is not an Azumaya algebra on $R$.  We define the \vocab{algebraic ramification index} $e = e(\Delta/R)$ to be the unique integer satisfying $J(A)^e = \pi J(A)$, and the \vocab{cohomological ramification index} $e' = e'(\Delta/R)$ to be $e' = \dim_k(Z(A/J(A)))$.  We define the \vocab{inertia degree} $f = f(\Delta/R$) to be the dimension of $A/J(A)$ over its center.
\end{defn}

Note that since all of the maximal $R$-orders in $\Delta$ are conjugate, whether or not $\Delta$ ramifies and the value of its ramification index is independent of the choice of maximal $R$-order in $\Delta$.
\begin{lem}
Let $A$ be a maximal $R$-order in $\Delta$, with algebraic and cohomological ramification index $e$ and $e'$, respectively, and with inertia degree $f$.  Then $ee'f^2 = n^2$.
\end{lem}
\begin{proof}
Since $A$ is a free $R$-module of rank $n^2$, we have
$$n^2 = \dim_{k}(\ol A) = e\dim_{k}(\ol A/J(\ol A)) = e\dim_k(A/J(A)) = ee'f^2.$$
\end{proof}

\begin{prop}
Let $A$ be a maximal $R$-order in $\Delta$, with algebraic and cohomological ramification $e$ and $e'$, respectively.  Then the following are equivalent
\begin{enumerate}[(a)]
\item  $\Delta$ is ramified over $R$
\item  $e>1$ or $e'>1$
\end{enumerate}
\end{prop}
\begin{proof}
If $\Delta$ is unramified over $R$, then $A$ is an Azumaya algebra and therefore $\ol A = A/\pi A$ is an Azumaya algebra on $k$.  In particular, this means $\ol A$ is a CSA over $k$, so that $J(\ol A) = 0$ and $Z(\ol A) = k$.  Since $A/J(A)\cong \ol A/J(\ol A)$, this implies that $A/J(A)\cong \ol A$.  Hence $\dim_k(Z(A/J(A))) = 1$ and $J(A) = \pi A$.  Therefore (b) implies (a).

If $\Delta$ is ramified over $R$, then $A$ is not Azumaya, and therefore $\ol A$ is not Azumaya.  If $e = 1$, then $\ol A$ is simple with center a nontrivial cyclic extension of $k$, and therefore $e'>1$.  Hence (a) implies (b).
\end{proof}

\begin{thm}
Suppose there is a finite extension $K'/K$ which is unramified over $R$ and splits $\Delta$.  Then the algebraic and cohomological ramification of $\Delta$ over $R$ are the same.
\end{thm}
\begin{proof}
See \cite{Artin&deJong}.
\end{proof}




