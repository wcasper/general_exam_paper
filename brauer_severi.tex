\section{Brauer-Severi Schemes}
Let $X$ be a proper integral scheme over an algebraically closed field $k$.  We have already pointed out that the set $H^1(X,\bbpgl_n)$ classifies twisted forms of two different algebraic structures over $X$:
$$\xymatrix{
& \ar@{<->}[dl]H^1(X,\bbpgl_n) \ar@{<->}[dr]&\\
\Az_n(X)\ar@{-->}[rr] & & \twist(\bbp_X^{n-1})
}$$
Hence rank $n$ Azumaya algebras on $X$ and twisted forms of $\bbp^{n_1}_X$ are in bijection.  One way of explicitly obtaining this bijection is by means of generating the Brauer-Severi scheme associated to an Azumaya algebra, which we now define.

Let $A$ a sheaf of $\sheaf O_X$-algebras, locally free as an $\sheaf O_X$-module.  We define a functor $\FBS_n(A): \Sch_X\rightarrow \Sets$ by
$$\FBS_n(A): (T\rightarrow X)\mapsto \{\text{quot. left $A_T$-modules of $A_T$ loc. free of rank $n^2-n$}\}.$$
We will prove that this functor is representable by proving that it is a locally closed subfunctor of the Quot functor.  The latter is representable, since $X$ is proper over $k$ \cite{nitsure2005construction}.
\begin{lem}
The functor $\FBS_n(A)$ is representable by a quasi-projective $k$-scheme.
\end{lem}
\begin{proof}
Define a subfunctor $\FQ_n$ of the Quot functor $\FQ := \FQuot_{A/X/k}$ by
$$\FQ_n: (T\rightarrow X)\mapsto \{(\sheaf F,q)\in \FQ(T): \sheaf F\ \text{is locally free of rank $n$}\}.$$
We claim that $\FQ_n$ is an open subfunctor of $\FQ$.  To see this, suppose that $T\in\Sch_k$ and that we have a natural transformation $h_T\rightarrow \FQ$.  Since $\FQ$ is represented by a scheme $Q$, there exists a natural isomorphism $\eta: \Hom(-,Q)\rightarrow\FQ$.  Furthermore, Yoneda tells us $h_T\rightarrow \FQ$ corresponds to a unique morphism $f:T\rightarrow Q$, thereby a unique element $(\sheaf F,q) := \eta(T)(f)\in \FQ(T)$.  Note that since $\sheaf F$ is flat over $T$, it is locally free over $T$.  Let $U\subseteq T$ be the set of points $t\in T$ for which $\sheaf F_t$ has rank $n$.  Then $U$ is an open subset of $T$.  Thus to prove that $\FQ_n$ is an open subfunctor of $\FQ$ it suffices to show that $\FQ_n\times_{\FQ}h_T = h_U$.

To see this, suppose that we have a commutative diagram
$$\xymatrix{
h_{T'}\ar[d]\ar[r] & \FQ_n\ar[d]\\
h_{T}\ar[r]        & \FQ
}.$$
Then we have a natural transformation $h_{T'}\rightarrow\FQ$, corresponding to a morphism $f': T'\rightarrow Q$, and determining an element $(\sheaf F',q') := \eta(T')(f')\in \FQ(T')$.  Since $h_{T'}\rightarrow\FQ$ factors through $\FQ_n$, this tells us $\sheaf F'$ is locally free of rank $n$.  Since $h_{T'}\rightarrow\FQ$ factors through $h_T$, this tells us that $f'$ factors through $f$; meaning that there is a morphism $g:T'\rightarrow T$ with $f'=gf$.  By functorality, it follows that $\eta(T)(f') = \eta(g)(f) = (g^*\sheaf F,g^*q)$.  Thus $g^*\sheaf F$ is locally free of rank $n$.  Since pulling back locally free modules doesn't affect the rank, the image of $g$ must lie in $U$.  Hence $T'\rightarrow T$ factors through $U\rightarrow T$, giving us a unique morphism $h_{T'}\rightarrow h_U$.  Thus $h_U$ satisfies the universal property of the pullback, and this proves our claim.  In particular, this shows that $\FQ_n$ is representable by an open subscheme of $Q$.

Next, define a subfunctor $\FQ'$ of $\FQ$ by
$$\FQ': (T\rightarrow X)\mapsto \{(\sheaf F,q)\in \FQ(T): \sheaf F\ \text{is a left $A$-module}\}.$$
We claim that $\FQ'$ is a locally closed subfunctor of $\FQ$.  To see this, again consider a natural transformation $h_T\rightarrow\FQ$, let $(\sheaf F,q)$ be the corresponding element of $\FQ(T)$, and let $I$ be the kernel of $A_T\rightarrow \sheaf F$.  We define
$$Z = \{t\in T: aI_t\subseteq I_t,\ \forall a\in A\}.$$
Then $Z$ is a locally closed subscheme of $Y$.  To check this, we may immediately reduce to the case where $T=\spec(R)$ for finitely generated $k$-algebra $R$ and $A$ is generated by a free $R$-module $A$.  Here, it's easy to express $A_\p I_\p\subseteq I_p$ in terms of finitely many relations on $R$.  By a similar argument to before, we prove that $h_Z\cong h_T\times_{\FQ} \FQ'$, and therefore that $\FQ'$ is locally closed subfunctor of $h_Z$.  In particular it is representable.

Noting that $\FBS_n(A)$ is the intersection of $\FQ'$ and $\FQ_n$, we see that $\FBS_n(A)$ is representable by a locally closed subscheme $\BS_n(A)$ of $Q$.  Since $Q$ is projective over $k$, this also shows us that $\BS_n(A)$ is quasiprojective over $k$.
\end{proof}

\begin{defn}
We denote the $X$-scheme representing the functor $\FBS_n(A)$ as $\BS_n(A)$, and call it the \vocab{Brauer-Severi scheme of $A$}.  In the special case that $X=\spec(K)$ for some finitely generated field extension $K/k$, and $\Delta$ is a central simple $K$-algebra, the Brauer-Severi scheme is also called the \vocab{Brauer-Severi variety}.
\end{defn}

Consider the sheaf of $\sheaf O_X$-algebras $M_n$ defined on the (big) etale site of $X$ by
$$M_n: (T\rightarrow X) = M_n(\Gamma(T,\sheaf O).$$
\begin{lem}
$\BS_n(M_n) = \bbp_X^{n-1}$
\end{lem}
\begin{proof}
Suppose that $I$ is a left ideal of $M_n(\sheaf O_T)$, locally free of rank $n$ as an $R$-module.  Then $I$ is also locally free as a left $M_n(\sheaf O_T)$-module since $\sheaf O$ and $M_n$ are Morita equivalent.  Noting that a principally generated free module of $I$ has rank $n$ as an $R$-module.  Hence $I$ is principal, and this shows that
\begin{align*}
\FBS_n(M_n)(T\rightarrow X)
  & = \{\text{left ideals of $M_n(\sheaf O_T)$ locally free over $T$ of rank $n$}\}\\
  & = \{\text{$M_n(\sheaf O_T)a$: $a$ not a zero divisor}\}\\
  & = \{\text{one-dimensional linear subspaces of $\bba^n_T$}\}\\
  & = \Gamma(\bbp^{n-1}_T,\sheaf O).
\end{align*}
Therefore $\FBS_n(M_n) = \bbp^{n-1}_X$ as sheaves on the big etale site of $X$, and consequently $\BS_n(M_n) = \bbp^{n-1}_X$.
\end{proof}

\begin{ex}\label{bs-ex0}
Let $k$ be a field and $\Delta = M_n(k)$.  Then $\BS_n(\Delta) \cong \bbp_k^{n-1}$.
\end{ex}

\begin{ex}\label{bs-ex1}
Let $k$ be a field, and $a,b\in k^*$ not squares, and $\Delta = (a,b)$ the quaternion algebra
$$(a,b) = k\langle x,y\rangle/(x^2-a,y^2-b,xy+yx).$$
Then $\BS_n(\Delta) = V(X^2-aY^2-bZ^2)\subseteq \bbp_k^2$.
\end{ex}

\begin{ex}\label{bs-ex2}
Let $k$ be a field, and $\Delta\subseteq M_2(k)$ be
$$\Delta = \left(\begin{array}{cc}
k & k\\
0 & k
\end{array}\right).$$
Then $\BS_n(\Delta) \cong \bbp^1_k\vee\bbp^1_k$
\end{ex}

\begin{lem}\label{bs-pullback lemma}
Let $f: Y\rightarrow X$ be a morphism of schemes.  Then
$$\BS_n(A)\times_X Y\cong \BS_n(f^*A).$$
\end{lem}
\begin{proof}
Let $F: \Sch_Y\rightarrow\Sch_X$ be the natural map induced by $X$.  Then for any $V\in\Sch_X$,
$$h_V\circ F = h_{V\times_XY}.$$
Thus to show that
$$\BS_n(A)\times_X Y\cong \BS_n(f^*A),$$
by Yoneda, it suffices to prove that
$$\FBS_n(A)\circ F = \FBS_n(f^*A).$$
Take $T\in\Sch_Y$ with $\pi: T\rightarrow Y$.  Then
\begin{align*}
\FBS_n(A)(F(T)) & = \{I\subseteq (f\pi)^*A: I\ \text{loc. free rk. $n$ ideal}\}\\
                     & = \{I\subseteq \pi^*f^*A: I\ \text{loc. free rk. $n$ ideal}\}\\
                     & = \FBS_n(f^*A)(T).
\end{align*}
\end{proof}

\begin{prop}
Let $A$ be an Azumaya algebra.  Then Brauer-Severi scheme $\BS_n(A)$ is etale-locally isomorphic to $\bbp_X^{n-1}$, and the functor $A\mapsto \BS_n(A)$ induces a bijective correspondence
$$\Az_n(X)\longleftrightarrow \{\text{iso classes of twisted forms of $\bbp^{n-1}_K$}\}.$$
\end{prop}
\begin{proof}
The fact that $\BS_n(A)$ is etale locally isomorphic to $\bbp^{n-1}_X$ follows from the previous lemma and the fact that $A$ is etale locally isomorphic to $M_n$.
\end{proof}


